\documentclass{article}
\usepackage[utf8]{inputenc}
\usepackage[T1]{fontenc}
\usepackage{babel}
\usepackage{float}
\usepackage{xcolor}  % kolory motywu
\usepackage{tikz}
\usetikzlibrary{angles}
\usetikzlibrary{quotes}
\usetikzlibrary{decorations.pathreplacing}
\usetikzlibrary{calligraphy}
\usetikzlibrary{arrows.meta}
\usetikzlibrary{calc}
\usepackage{pgfplots}
\usepackage{pgfplotstable}
\pgfplotsset{compat=1.9}
\usepackage{amsmath}  % równania
\usepackage{amssymb}
\usepackage{bbold}
\usepackage{physics2}  % pochodne, macierze itp
\usephysicsmodule{ab}
\usephysicsmodule{diagmat}
\usephysicsmodule{xmat}
\usephysicsmodule{nabla.legacy}
\usephysicsmodule{op.legacy}
\makeatletter
%\newcommand\vb[1]{\@ifstar\boldsymbol\mathbf{#1}}
\newcommand\vb[1]{\@ifstar\boldsymbol\mathbf{#1}}
\newcommand\va[1]{\@ifstar{\vec{#1}}{\vec{\mathrm{#1}}}}
\newcommand\vu[1]{%
	\@ifstar{\hat{\boldsymbol{#1}}}{\hat{\boldsymbol{#1}}}}
\makeatother
\usepackage{fixdif, derivative}  % pochodne
%\usepackage{mhchem}
\usepackage{siunitx}
\DeclareSIUnit{\nothing}{\relax}
\sisetup{parse-numbers = false}
\usepackage{booktabs}
\usepackage{tabularx}
\title{Prawa Newtona}
%\author{Rafał Staroszczyk}
\date{}

%\DeclareMathOperator{\Col}{Col}
%\DeclareMathOperator{\Nul}{Nul}
%\DeclareMathOperator{\arctg}{arctg}
%\DeclareMathOperator{\tgh}{tgh}


\setlength{\abovedisplayskip}{0pt}
\setlength{\belowdisplayskip}{0pt}
\setlength{\abovedisplayshortskip}{0pt}
\setlength{\belowdisplayshortskip}{0pt}

\newcommand{\inv}[1]{\frac{1}{#1}}

\usepackage{enumitem}
% Środowisko zadań
\usepackage{exercise}
\renewcommand{\ExerciseName}{Zadanie}
\renewcommand{\AnswerName}{Odpowiedź do zadania}
%\renewcommand{\ExerciseHeader}{\noindent{%
%\textbf{\large\ExerciseHeaderDifficulty\ExerciseName\ %
%\ExerciseHeaderNB\ExerciseHeaderTitle\ExerciseHeaderOrigin}}}
%\renewcommand{\AnswerHeader}{\noindent{\textbf{\AnswerName\ %
%\ExerciseHeaderNB}} }
\renewcommand{\theExePart}{(\alph{ExePart})\quad}
\renewcommand{\ExePartName}{}
\renewcommand{\ExePartHeader}{%
\indent{\ExePartHeaderDifficulty\ExePartName%
\ExePartHeaderNB\ExePartHeaderTitle}}
%\numberwithin{Exercise}{section}
%\numberwithin{Answer}{section}

\begin{document}
\maketitle
Tarcie jest siłą niekonserwatywną, która "opiera" się ruchowi. Wyróżniamy dwa rodzaje siły tarcia: statyczna, gdy ciała nie poruszają się względem siebie, oraz kinetyczna, gdy się poruszają. 
\section{Siła tarcia kinetycznego}
Wielkość siły tarcia kinetycznego zależy jedynie od siły nacisku prostopadłej do powierzchni oraz współczynnika tarcia kinetycznego:
\begin{equation}
T_{k} = \mu_{k} N.
\end{equation}
Wielkość ta nie zależy od pola powierzchni, ani prędkości. Jest zawsze skierowana przeciwnie do ruchu. 

\section{Siła tarcia statycznego}
Wystepuje w przypadku braku ruchu. Wynika z tego, że jest to również funkcja działającej siły. Siła tarcia statycznego opisana jest wzorem:
\begin{equation}
T_{s} \leq T_{s,max} = \mu_{s} N,
\end{equation}
gdzie $T_{s.max}$ to maksymalna wartość siły tarcia, po przekroczeniu której rozpoczyna się ruch. Jest skierowana przeciwnie do działającej siły zewnętrznej. 
\end{document}