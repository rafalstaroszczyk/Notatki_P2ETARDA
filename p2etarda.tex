\documentclass[xcolor]{beamer}
\usepackage[utf8]{inputenc}
\usepackage[T1]{fontenc}
\usepackage{babel}
\usepackage{float}
\usepackage{xcolor}  % kolory motywu
\usepackage{tikz}
\usetikzlibrary{angles}
\usetikzlibrary{quotes}
\usetikzlibrary{decorations.pathreplacing}
\usetikzlibrary{calligraphy}
\usetikzlibrary{arrows.meta}
\usetikzlibrary{calc}
\usepackage{pgfplots}
\usepackage{pgfplotstable}
\pgfplotsset{compat=1.9}
\usepackage{amsmath}  % równania
\usepackage{amssymb}
\usepackage{bbold}
\usepackage{physics2}  % pochodne, macierze itp
\usephysicsmodule{ab}
\usephysicsmodule{diagmat}
\usephysicsmodule{xmat}
\usephysicsmodule{nabla.legacy}
\usephysicsmodule{op.legacy}
\usefonttheme[onlymath]{serif}
\makeatletter
%\newcommand\vb[1]{\@ifstar\boldsymbol\mathbf{#1}}
\newcommand\vb[1]{\@ifstar\boldsymbol\mathbf{#1}}
\newcommand\va[1]{\@ifstar{\vec{#1}}{\vec{\mathrm{#1}}}}
\newcommand\vu[1]{%
	\@ifstar{\hat{\boldsymbol{#1}}}{\hat{\boldsymbol{#1}}}}
\makeatother
%\usepackage{fixdif, derivative}  % pochodne
%\usepackage{mhchem}
\usepackage{siunitx}
\DeclareSIUnit{\nothing}{\relax}
\sisetup{parse-numbers = false}
\usepackage{booktabs}
\usepackage{tabularx}
%\title{Hertz Vector Potentials}
%\author{Rafał Staroszczyk}
%\date{}
\usetheme{Hannover}
\usecolortheme{spruce}

%\DeclareMathOperator{\Col}{Col}
%\DeclareMathOperator{\Nul}{Nul}
%\DeclareMathOperator{\arctg}{arctg}
%\DeclareMathOperator{\tgh}{tgh}


\setlength{\abovedisplayskip}{0pt}
\setlength{\belowdisplayskip}{0pt}
\setlength{\abovedisplayshortskip}{0pt}
\setlength{\belowdisplayshortskip}{0pt}

\newcommand{\inv}[1]{\frac{1}{#1}}

\begin{document}
\section{Jednostki}
\begin{frame}{Układ SI}
	\begin{table}[H]
		\centering
		\begin{tabularx}{\textwidth}{XX}
			\toprule
			Wielkość podstawowa & Jednostka \\
			\midrule
			Długość & metr (\unit{\m}) \\
			Masa & kilogram (\unit{\kg}) \\
			Czas & sekunda (\unit{\s}) \\
			Natężenie prądu & amper (\unit{\A}) \\
			Temperatura termodynamiczna & kelwin (\unit{\K}) \\
			Liczność materii & mol (\unit{\mol}) \\
			Światłość & kandela (\unit{\candela}) \\
			\bottomrule
		\end{tabularx}
	\end{table}
\end{frame}

\begin{frame}{Przedrostki SI}
		\begin{table}[H]
		\centering
		\begin{tabularx}{\textwidth}{XXXX}
			\toprule
			Przedrostek & Mnożnik & Przedrostek & Mnożnik \\
			\midrule
			peta (\unit{\peta\nothing}) & $10^{15}$ & decy (\unit{\deci\nothing}) & $10^{-1}$ \\
			tera (\unit{\tera\nothing}) & $10^{12}$ & centy (\unit{\centi\nothing}) & $10^{-2}$ \\
			giga (\unit{\giga\nothing}) & $10^{9}$ & mili (\unit{\milli\nothing}) & $10^{-3}$ \\
			mega (\unit{\mega\nothing}) & $10^{6}$ & mikro (\unit{\micro\nothing}) & $10^{-6}$ \\
			kilo (\unit{\kilo\nothing}) & $10^{3}$ & nano (\unit{\nano\nothing}) & $10^{-9}$ \\
			hekto (\unit{\hecto\nothing}) & $10^{2}$ & piko (\unit{\pico\nothing}) & $10^{-12}$ \\
			deka (\unit{\deca\nothing}) & $10^{1}$ & femto (\unit{\femto\nothing}) & $10^{-15}$ \\
			& $10^{0}$ & atto (\unit{\atto\nothing}) & $10^{-18}$ \\
			\bottomrule
		\end{tabularx}
	\end{table}
\end{frame}

\begin{frame}
	\begin{block}{}
		\begin{equation*}
			\qty{6}{\km} = ?\unit{\cm}
		\end{equation*}
	\end{block}
\end{frame}

\begin{frame}
	\begin{block}{}
		\begin{equation*}
			\qty{6}{\km} = \qty{6}{\km}*\frac{\qty{10^{3}}{\m}}{\qty{1}{\km}}*\frac{\qty{10^{2}}{\cm}}{\qty{1}{\m}} = \qty{6*10^{5}}{\cm}
		\end{equation*}
	\end{block}
\end{frame}

\begin{frame}
	\begin{block}{}
		\begin{equation*}
			\qty{9}{\km} = ?\unit{\cm}
		\end{equation*}
	\end{block}
\end{frame}

\section{Wektory}
\section{Kinematyka}
\section{Prawa Newtona}
\section{Tarcie}
\section{Zasada zachowania energii}
\section{Zasada zachowania pędu}
\section{Ruch obrotowy}
\section{Statyka}
\end{document}