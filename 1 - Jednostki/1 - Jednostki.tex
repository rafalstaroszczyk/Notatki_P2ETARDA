\documentclass{article}
\usepackage[utf8]{inputenc}
\usepackage[T1]{fontenc}
\usepackage{babel}
\usepackage{float}
\usepackage{xcolor}  % kolory motywu
\usepackage{tikz}
\usetikzlibrary{angles}
\usetikzlibrary{quotes}
\usetikzlibrary{decorations.pathreplacing}
\usetikzlibrary{calligraphy}
\usetikzlibrary{arrows.meta}
\usetikzlibrary{calc}
\usepackage{pgfplots}
\usepackage{pgfplotstable}
\pgfplotsset{compat=1.9}
\usepackage{amsmath}  % równania
\usepackage{amssymb}
\usepackage{bbold}
\usepackage{physics2}  % pochodne, macierze itp
\usephysicsmodule{ab}
\usephysicsmodule{diagmat}
\usephysicsmodule{xmat}
\usephysicsmodule{nabla.legacy}
\usephysicsmodule{op.legacy}
\makeatletter
%\newcommand\vb[1]{\@ifstar\boldsymbol\mathbf{#1}}
\newcommand\vb[1]{\@ifstar\boldsymbol\mathbf{#1}}
\newcommand\va[1]{\@ifstar{\vec{#1}}{\vec{\mathrm{#1}}}}
\newcommand\vu[1]{%
	\@ifstar{\hat{\boldsymbol{#1}}}{\hat{\boldsymbol{#1}}}}
\makeatother
%\usepackage{fixdif, derivative}  % pochodne
%\usepackage{mhchem}
\usepackage{siunitx}
\DeclareSIUnit{\nothing}{\relax}
\DeclareSIUnit{\ly}{ly}
\sisetup{parse-numbers = false}
\usepackage{booktabs}
\usepackage{tabularx}
\title{Układ jednostek SI}
%\author{Rafał Staroszczyk}
\date{}

%\DeclareMathOperator{\Col}{Col}
%\DeclareMathOperator{\Nul}{Nul}
%\DeclareMathOperator{\arctg}{arctg}
%\DeclareMathOperator{\tgh}{tgh}


\setlength{\abovedisplayskip}{0pt}
\setlength{\belowdisplayskip}{0pt}
\setlength{\abovedisplayshortskip}{0pt}
\setlength{\belowdisplayshortskip}{0pt}

\newcommand{\inv}[1]{\frac{1}{#1}}

\usepackage{enumitem}
% Środowisko zadań
\usepackage{exercise}
\renewcommand{\ExerciseName}{Zadanie}
\renewcommand{\AnswerName}{Odpowiedź do zadania}
%\renewcommand{\ExerciseHeader}{\noindent{%
%\textbf{\large\ExerciseHeaderDifficulty\ExerciseName\ %
%\ExerciseHeaderNB\ExerciseHeaderTitle\ExerciseHeaderOrigin}}}
%\renewcommand{\AnswerHeader}{\noindent{\textbf{\AnswerName\ %
%\ExerciseHeaderNB}} }
\renewcommand{\theExePart}{(\alph{ExePart})\quad}
\renewcommand{\ExePartName}{}
\renewcommand{\ExePartHeader}{%
\indent{\ExePartHeaderDifficulty\ExePartName%
\ExePartHeaderNB\ExePartHeaderTitle}}
%\numberwithin{Exercise}{section}
%\numberwithin{Answer}{section}

\begin{document}
\maketitle
\section{Jednostki podstawowe}
W układzie wykorzystuje się 7 jednostek podstawowych:
\begin{table}[H]
\centering
\begin{tabular}{rl}
\toprule
Wielkość podstawowa & Jednostka podstawowa \\
\midrule
długość & metr (\unit{\m}) \\
masa & kilogram (\unit{\kg}) \\
czas & sekunda (\unit{\s}) \\
natężenie prądu elektrycznego & amper (\unit{\A}) \\
temperatura termodynamiczna & kelwin (\unit{\K}) \\
liczność materii & mol (\unit{\mol}) \\
światłość & kandela (\unit{\candela}) \\
\bottomrule
\end{tabular}
\caption{Jednostki podstawowe}
\end{table}
Jednostki takie jak amper i kelwin zapisuje się z małej litery według wymowy, a nie jak nazwiska Amp\`{e}re oraz Kelvin. Symbole się nie odmieniają $\qty{2}{\mol}$, ale czyta się $2$ mole.

\section{Jednostki pochodne}
Z jednostek podstawowych można skonstruować jednostki pochodne, na przykład:
\begin{table}[H]
\centering
\begin{tabular}{rl}
\toprule
Wielkość podstawowa & Jednostka podstawowa \\
\midrule
miara kąta płaskiego & radian (\unit{\radian=\m\per\m}) \\
miara kąta bryłowego & steradian (\unit{\steradian=\m\squared\per\m\squared}) \\
częstotliwość & herc (\unit{\Hz=\s^{-1}}) \\
siła & niuton (\unit{\newton=\kg.\m.\s^{-2}}) \\
ciśnienie & paskal (\unit{\pascal=\newton\per\metre\squared=\kg.\m^{-1}.\s^{-2}}) \\
energia & dżul (\unit{\J=\newton.\m=\kg.\m^{2}.\s^{-2}}) \\
moc & wat (\unit{\watt=\joule\per\s=\kg.\m^{2}.\s^{-3}}) \\
napięcie elektryczne & wolt (\unit{\volt=\joule\per\coulomb=\kg.\m^{2}.\s^{-3}.\A^{-1}}) \\
strumień świetlny & lumen (\unit{\lumen=\candela.\steradian=\candela}) \\
natężenie oświetlenia & luks (\unit{\lux=\lumen\per\m\squared=\candela.\m^{-2}}) \\
aktywność promieniotwórcza & bekerel (\unit{\becquerel=\s^{-1}}) \\
\bottomrule
\end{tabular}
\caption{Przykładowe jednostki pochodne}
\end{table}
Jednostki, które wyglądają na identyczne, np. \unit{\hertz} oraz \unit{\becquerel}, w rzeczywistości nie są, ponieważ występują jako miara innych procesów. \unit{\hertz} jako ilość cykli w fali, która jest zjawiskiem ciągłym, a \unit{\becquerel} jako ilość rozpadów promieniotwórczych na sekundę, co jest zjawiskiem losowym.

\section{Przedrostki układu SI}
Do każdej jednostki podstawowej oraz pochodnej można dodać przedrostek oznaczający potęgę $10$. 
\begin{table}[H]
\centering
\begin{tabularx}{\textwidth}{XXXX}
\toprule
Przedrostek & Mnożnik & Przedrostek & Mnożnik \\
\midrule
deka (\unit{\deca\nothing}) & $10^{1}$ & decy (\unit{\deci\nothing}) & $10^{-1}$ \\
hekto (\unit{\hecto\nothing}) & $10^{2}$ & centy (\unit{\centi\nothing}) & $10^{-2}$ \\
kilo (\unit{\kilo\nothing}) & $10^{3}$ & mili (\unit{\milli\nothing}) & $10^{-3}$ \\
mega (\unit{\mega\nothing}) & $10^{6}$ & mikro (\unit{\micro\nothing}) & $10^{-6}$ \\
giga (\unit{\giga\nothing}) & $10^{9}$ & nano (\unit{\nano\nothing}) & $10^{-9}$ \\
tera (\unit{\tera\nothing}) & $10^{12}$ & piko (\unit{\pico\nothing}) & $10^{-12}$ \\
peta (\unit{\peta\nothing}) & $10^{15}$ & femto (\unit{\femto\nothing}) & $10^{-15}$ \\
eksa (\unit{\exa\nothing}) & $10^{18}$ & atto (\unit{\atto\nothing}) & $10^{-18}$ \\
zetta (\unit{\zetta\nothing}) & $10^{21}$ & zepto (\unit{\zepto\nothing}) & $10^{-21}$ \\
jotta (\unit{\yotta\nothing}) & $10^{24}$ & jokto (\unit{\yocto\nothing}) & $10^{-24}$ \\
ronna (\unit{\ronna\nothing}) & $10^{27}$ & ronto (\unit{\ronto\nothing}) & $10^{-27}$ \\
quetta (\unit{\quetta\nothing}) & $10^{30}$ & quecto (\unit{\quecto\nothing}) & $10^{-30}$ \\
\bottomrule
\end{tabularx}
\caption{Przedrostki układu SI}
\end{table}
W każdej jednostce można użyć tylko jednego przedrostka. Aby uniknąć pomyłek powinno się pisać przedrostek razem z jednostką, a każde jednostki oddzielone, \unit{\milli\second} to milisekunda, a \unit{\metre.\second} to metr razy sekunda. W przypadku potęg przedrostek odnosi się do niepotęgowanej jednostki, \unit{\femto\metre\squared} jest równy $\qty{10^{-30}}{\metre\squared}$, a nie $\qty{10^{-15}}{\metre\squared}$.

\section{Przykładowe zadania}\setlength{\parindent}{0pt}
\begin{Exercise}[number={24}]
\ExePart $\qty{980}{\peta\second}$;
\ExePart $\qty{980}{\femto\second}$;
\ExePart $\qty{17}{\nano\second}$;
\ExePart $\qty{577}{\micro\second}$.
\end{Exercise}

\begin{Answer}
\begin{enumerate}[label={(\alph*)}, noitemsep]
\item $\qty{980}{\peta\second} = \qty{980*10^{15}}{\second} = \qty{9,80*10^{17}}{\second}$; 
\item $\qty{980}{\femto\second} = \qty{980*10^{-15}}{\second} = \qty{9,80*10^{-13}}{\second}$;
\item $\qty{17}{\nano\second} = \qty{17*10^{-9}}{\second} = \qty{1,7*10^{-8}}{\second}$;
\item $\qty{577}{\micro\second} = \qty{577*10^{-6}}{\second} = \qty{5,77*10^{-4}}{\second}$.
\end{enumerate}
\end{Answer}

\begin{Exercise}[number={25}]
\ExePart $\qty{9,57*10^{5}}{\second}$;
\ExePart $\qty{0,045}{\second}$;
\ExePart $\qty{5,5*10^{-7}}{\second}$;
\ExePart $\qty{3,16*10^{7}}{\second}$.
\end{Exercise}

\begin{Answer}
\begin{enumerate}[label={(\alph*)}, noitemsep]
\item $\qty{9,57*10^{5}}{\second} = \qty{957*10^{3}}{\second} = \qty{957}{\kilo\second}$;
\item $\qty{0,045}{\second} = \qty{45*10^{-3}}{\second} = \qty{45}{\milli\second}$;
\item $\qty{5,5*10^{-7}}{\second} = \qty{550*10^{-9}}{\second} = \qty{550}{\nano\second}$;
\item $\qty{3,16*10^{7}}{\second} = \qty{31,6*10^{6}}{\second} = \qty{31,6}{\mega\second}$.
\end{enumerate}
\end{Answer}

\begin{Exercise}[number={26}]
\ExePart $\qty{89}{\tera\metre}$;
\ExePart $\qty{89}{\pico\metre}$;
\ExePart $\qty{711}{\milli\metre}$;
\ExePart $\qty{0,45}{\micro\metre}$.
\end{Exercise}

\begin{Answer}
\begin{enumerate}[label={(\alph*)}, noitemsep]
\item $\qty{89}{\tera\metre} = \qty{89*10^{12}}{\metre} = \qty{8,9*10^{13}}{\metre}$;
\item $\qty{89}{\pico\metre} = \qty{89*10^{-12}}{\metre} = \qty{8,9*10^{-11}}{\metre}$;
\item $\qty{711}{\milli\metre} = \qty{711*10^{-3}}{\metre} = \qty{7,11*10^{-1}}{\metre}$;
\item $\qty{0,45}{\micro\metre} = \qty{0,45*10^{-6}}{\metre} = \qty{4,5*10^{-7}}{\metre}$.
\end{enumerate}
\end{Answer}

\begin{Exercise}[number={30}]
\ExePart $\qty{10*{21}}{\m^{3}}$ w $\unit{\km^{3}}$;
\ExePart $\qty{10*{21}}{\m^{3}}$ w $\unit{\cm^{3}}$.
\end{Exercise}

\begin{Answer}
\begin{enumerate}[label={(\alph*)}, noitemsep]
\item $\qty{10*{21}}{\m^{3}} = \qty{10^{21}}{(10^{-3}\km)^{3}} = \qty{10^{21}*10^{-9}}{\km^{3}} = \qty{10^{12}}{\km^{3}}$;
\item $\qty{10*{21}}{\m^{3}} = \qty{10^{21}}{(10^{2}\cm)^{3}} = \qty{10^{21}*10^{6}}{\cm^{3}} = \qty{10^{27}}{\cm^{3}}$.
\end{enumerate}
\end{Answer}

\begin{Exercise}[number={32}]
$\qty{33}{\m/\s}$ w $\unit{\km/\hour}$.
\end{Exercise}

\begin{Answer}
$\qty{33}{\m/\s} = \qty{33}{\frac{10^{-3}\km}{\inv{3600}\hour}} = \qty{33*\frac{3600}{1000}}{\km/\hour} = \qty{118,8}{\km/\hour}$.
\end{Answer}

\begin{Exercise}[number={41}]
$\qty{10*{18}}{\kg/m^{3}}$ w $\unit{\mega\gram/\micro\litre}$.
\end{Exercise}

\begin{Answer}
$\qty{1}{\kg} = \qty{10^{3}}{\gram} = \qty{10^{-3}*10^{6}}{\gram} = \qty{10^{-3}}{\mega\gram}$ \\
$\qty{1}{\m^{3}} = \qty{10^{6}}{\cm^{3}} = \qty{10^{6}}{\milli\litre} = \qty{10^{3}}{\litre} = \qty{10^{9}}{\micro\litre}$ \\
$\qty{10^{18}}{\kg/m^{3}} = \qty{10^{18}}{\frac{10^{-3}\mega\gram}{10^{9}\micro\litre}} = \qty{10^{6}}{\mega\gram/\micro\litre}$
\end{Answer}

\section{Jednostki poza układem SI, ale powszechne}
\begin{table}[H]
\centering
\begin{tabularx}{\textwidth}{XXX}
\toprule
Jednostka & Symbol & Wartość w SI \\
\midrule
angstrem & $\unit{\angstrom}$ & $\qty{10^{-10}}{\meter}$ \\
jednostka astronomiczna & $\unit{\astronomicalunit}$ & $\qty{149597870700}{\meter}$ \\
rok świetlny & $\unit{\ly}$ & $\qty{9,4607 * 10^{15}}{\meter}$ \\
tona & $\unit{\tonne}$ & $\qty{10^{3}}{\kg}$ \\
jednostka masy atomowej (dalton) & $\unit{\atomicmassunit} = \unit{\dalton}$ & $\qty{1,660539\ldots}{\kg}$ \\

\bottomrule
\end{tabularx}
\caption{Jednostki pozaukładowe}
\end{table}
\end{document}