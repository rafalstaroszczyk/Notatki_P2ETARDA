\documentclass{article}
\usepackage[utf8]{inputenc}
\usepackage[T1]{fontenc}
\usepackage{babel}
\usepackage{float}
\usepackage{xcolor}  % kolory motywu
\usepackage{tikz}
\usetikzlibrary{angles}
\usetikzlibrary{quotes}
\usetikzlibrary{decorations.pathreplacing}
\usetikzlibrary{calligraphy}
\usetikzlibrary{arrows.meta}
\usetikzlibrary{calc}
\usepackage{pgfplots}
\usepackage{pgfplotstable}
\pgfplotsset{compat=1.9}
\usepackage{amsmath}  % równania
\usepackage{amssymb}
\usepackage{bbold}
\usepackage{physics2}  % pochodne, macierze itp
\usephysicsmodule{ab}
\usephysicsmodule{diagmat}
\usephysicsmodule{xmat}
\usephysicsmodule{nabla.legacy}
\usephysicsmodule{op.legacy}
\makeatletter
%\newcommand\vb[1]{\@ifstar\boldsymbol\mathbf{#1}}
\newcommand\vb[1]{\@ifstar\boldsymbol\mathbf{#1}}
\newcommand\va[1]{\@ifstar{\vec{#1}}{\vec{\mathrm{#1}}}}
\newcommand\vu[1]{%
	\@ifstar{\hat{\boldsymbol{#1}}}{\hat{\boldsymbol{#1}}}}
\makeatother
\usepackage{fixdif, derivative}  % pochodne
%\usepackage{mhchem}
\usepackage{siunitx}
\DeclareSIUnit{\nothing}{\relax}
\sisetup{parse-numbers = false}
\usepackage{booktabs}
\usepackage{tabularx}
\title{Energia}
%\author{Rafał Staroszczyk}
\date{}

%\DeclareMathOperator{\Col}{Col}
%\DeclareMathOperator{\Nul}{Nul}
%\DeclareMathOperator{\arctg}{arctg}
%\DeclareMathOperator{\tgh}{tgh}
\DeclareMathOperator{\tg}{tg}
\DeclareMathOperator{\ctg}{ctg}


\setlength{\abovedisplayskip}{0pt}
\setlength{\belowdisplayskip}{0pt}
\setlength{\abovedisplayshortskip}{0pt}
\setlength{\belowdisplayshortskip}{0pt}

\newcommand{\inv}[1]{\frac{1}{#1}}

\usepackage{enumitem}
% Środowisko zadań
\usepackage{exercise}
\renewcommand{\ExerciseName}{Zadanie}
\renewcommand{\AnswerName}{Odpowiedź do zadania}
%\renewcommand{\ExerciseHeader}{\noindent{%
%\textbf{\large\ExerciseHeaderDifficulty\ExerciseName\ %
%\ExerciseHeaderNB\ExerciseHeaderTitle\ExerciseHeaderOrigin}}}
%\renewcommand{\AnswerHeader}{\noindent{\textbf{\AnswerName\ %
%\ExerciseHeaderNB}} }
\renewcommand{\theExePart}{(\alph{ExePart})\quad}
\renewcommand{\ExePartName}{}
\renewcommand{\ExePartHeader}{%
\indent{\ExePartHeaderDifficulty\ExePartName%
\ExePartHeaderNB\ExePartHeaderTitle}}
%\numberwithin{Exercise}{section}
%\numberwithin{Answer}{section}

\begin{document}
\maketitle
\section{Praca}
Praca zdefiniowana jest poprzez całkę krzywoliniową:
\begin{equation}\label{eq:praca_def}
W\pab{A \rightarrow B} = \int_{A}^{B} \vb{F}\cdot\d{\vb{r}},
\end{equation}
która zależy w ogólności nie tylko od punktów krańcowych, ale też drogi. Aby obliczyć całkę krzywoliniową należy wprowadzić parametr, poprzez który wyrażamy współrzędne. Parametr ten opisuje tor ruchu. Jeśli parametr ten jest drogą $s$, a $\theta$ to kąt między siłą $\vb{F}$ a styczną do toru w kierunku wzrostu $s$.
\begin{equation*}
W\pab{A \rightarrow B} = \int_{s_A}^{s_B} F_{\parallel}\cos\theta\d{s}.
\end{equation*}
Zapisując wzór \eqref{eq:praca_def} w innej postaci wykorzystując $\d{\vb{r}} = \vb{v}\d{t}$ oraz $\vb{F} = m\odv{\vb{v}}{t}$:
\begin{equation}\label{eq:praca_kinetyczna}
\begin{split}
W\pab{A \rightarrow B} &= \int_{A}^{B} m\odv{\vb{v}}{t}\cdot\vb{v}\d{t} = \\
&= \inv{2}\int_{A}^{B} m\odv{v^2}{t}\d{t} = \\
&= \int_{A}^{B} \odv{\inv{2}mv^2}{t}\d{t}.
\end{split}
\end{equation}

\section{Energia kinetyczna}
Wprowadzamy nową wielkość:
\begin{equation}\label{eq:kinetyczna_def}
T = \frac{mv^2}{2} = \frac{p^2}{2m}.
\end{equation}
Wykorzystując tą wielkość otrzymujemy z \eqref{eq:praca_kinetyczna}:
\begin{equation}
W\pab{A \rightarrow B} = T_{B} - T_{A}.
\end{equation}
Energia kinetyczna jest wielkością skalarną.

\section{Energia potencjalna}
W wielu przypadkach praca \eqref{eq:praca_def} nie zależy od drogi całkowania. Takie siły nazywa się zachowawczymi lub konserwatywnymi. Przykładem takiej siły jest siła grawitacji pola jednorodnego $\vb{F} = m\vb{g}; \quad g=const.$ oraz pola centralnego $\vb{F} = - G \frac{mM}{r^2} \vu{r}$. Przykładem siły, która nie jest konserwatywna jest siła tarcia. \par 
Dla sił konserwatywnych możemy je zapisać jako:
\begin{equation}
\vb{F}_{zach} = - \grad U,
\end{equation}
gdzie $U$ to energia potencjalna. W przypadku jednowymiarowym wzór ten ma postać:
\begin{equation}
F_{zach} = - \odv{U}{x}.
\end{equation}
Praca \eqref{eq:praca_def} w przypadku sił zachowawczych sprowadza się do:
\begin{equation}\label{eq:praca_potencjalna_zach}
W\pab{A \rightarrow B} = \int_{A}^{B} - \pab{\grad U} \cdot \d{\vb{r}} = - \pab{U_{B} - U_{A}} = U_{A} - U_{B}.
\end{equation}
Porównując \eqref{eq:praca_kinetyczna} oraz \eqref{eq:praca_potencjalna_zach} mamy:
\begin{gather}
W\pab{A \rightarrow B} = T_{B} - T_{A} = U_{A} - U_{B} \\
T_{A} + U_{A} = T_{B} + U_{B} = E = const.
\end{gather}
Suma energii kinetycznej $T$ oraz potencjalnej $U$ to całkowita energia mechaniczna $E$ i w przypadku jedynie sił konserwatywnych jest wielkością zachowaną. \par
W polu jednorodnym siły grawitacji $\vb{F} = -mg\vu{k}$:
\begin{equation}
\begin{split}
W\pab{A \rightarrow B} &= \int_{A}^{B} -mg \vu{k}\cdot\d{\vb{r}} = \\
&= -mg \int_{A}^{B} \vu{k}\cdot\d{\vb{r}} = -mgh.
\end{split}
\end{equation}
Jeśli wybierzemy poziom odniesienia taki, że $U_{A} = 0$:
\begin{gather}
U_{A} - U_{B} = -mgh, \\
U_{B} = mgh.
\end{gather}

\section{Przykład: ciało na równi pochyłej}
\subsection{Poprzez prawa Newtona}
Efektywna siła działająca na ciało to suma siły ciężkości $\vb{Q}$ oraz siły reakcji podłoża $\vb{R}$, która jest równa co do wartości $F_s = m g \sin\alpha$. Jeśli wysokość równi jest $h$, to droga przebyta przez ciało jest równa $s = \frac{h}{\sin\alpha}$. Jaka jest prędkość na końcu równi, jeśli na początku $v_0 = 0$? 
\begin{equation*}
\begin{cases}
v &= v_0 + g t \sin\alpha, \\
s &= v_0 t + \inv{2} g t^2 \sin\alpha.
\end{cases}
\end{equation*}
\begin{equation*}
\begin{cases}
v &= g t \sin\alpha, \\
t &= \sqrt{\frac{2s}{g \sin\alpha}} = \inv{\sin\alpha}\sqrt{\frac{2h}{g}}.
\end{cases}
\end{equation*}
\begin{equation}
v = g \inv{\sin\alpha}\sqrt{\frac{2h}{g}} \sin\alpha = \sqrt{2gh}.
\end{equation}

\subsection{Poprzez zasadę zachowania energii}
W stanie początkowym:
\begin{align}
T_{A} &= 0, \\
U_{A} &= mgh.
\end{align}
W stanie końcowym:
\begin{align}
T_{B} = \inv{2}mv^2, \\
U_{B} = 0.
\end{align}
Z zasady zachowania energii:
\begin{equation}
\begin{split}
0 + mgh = \inv{2}mv^2 + 0, \\
v = \sqrt{2gh}.
\end{split}
\end{equation}

\section{Praca w przypadku sił niezachowawczych}
Dzieląc siłę na zachowawcze i niezachowawcze otrzymujemy:
\begin{equation}\label{eq:praca_potencjalna_niezach}
\begin{split}
W\pab{A \rightarrow B} = \int_{A}^{B} \vb{F}\cdot\d{\vb{r}} = \\
= \int_{A}^{B} \vb{F}_{zach}\cdot\d{\vb{r}} + \int_{A}^{B} \vb{F}_{niezach}\cdot\d{\vb{r}} = \\
= \int_{A}^{B} - \pab{\grad U}\cdot\d{\vb{r}} + \int_{A}^{B} \vb{F}_{niezach}\cdot\d{\vb{r}} = \\
= U_{A} - U_{B} + W_{niezach}\pab{A \rightarrow B}
\end{split}
\end{equation}
Porównując \eqref{eq:praca_kinetyczna} oraz \eqref{eq:praca_potencjalna_niezach}:
\begin{gather}
W\pab{A \rightarrow B} = T_{B} - T_{A} = U_{A} - U_{B} + W_{niezach}\pab{A \rightarrow B} \\
\pab{T_{B} + U_{B}} - \pab{T_{A} + U_{A}} = W_{niezach}\pab{A \rightarrow B} \\
E_{B} - E_{A} = W_{niezach}\pab{A \rightarrow B}.
\end{gather}
W tym energia mechaniczna nie jest zachowana. 

\section{Przykład: ciało na równi pochyłej z tarciem}
Sytuacja podobna jak bez tarcia, ale działa dodatkowa siła:
\begin{equation*}
F_T = \mu_{k} N = \mu_{k}mg \cos\alpha.
\end{equation*}
\subsection{Poprzez prawa Newtona}
Siła wypadkowa:
\begin{equation}
F_{wyp} = mg \sin\alpha - \mu_{k}mg \cos\alpha.
\end{equation}
Przyspieszenie:
\begin{equation}
a = g \sin\alpha - \mu_{k}g \cos\alpha = g \pab{\sin\alpha - \mu_{k} \cos\alpha}.
\end{equation}
\begin{equation*}
\begin{cases}
v = v_0 + g t \pab{\sin\alpha - \mu_{k} \cos\alpha}, \\
s = v_0 t + \inv{2} g t^2 \pab{\sin\alpha - \mu_{k} \cos\alpha}.
\end{cases}
\end{equation*}
\begin{equation*}
\begin{cases}
v = g t \pab{\sin\alpha - \mu_{k} \cos\alpha}, \\
t = \sqrt{\frac{2s}{g \pab{\sin\alpha - \mu_{k} \cos\alpha}}} = \sqrt{\frac{2h}{g \sin\alpha \pab{\sin\alpha - \mu_{k} \cos\alpha}}}.
\end{cases}
\end{equation*}
\begin{equation}
\begin{split}
v = g \sqrt{\frac{2h}{g \sin\alpha \pab{\sin\alpha - \mu_{k} \cos\alpha}}} \pab{\sin\alpha - \mu_{k} \cos\alpha} = \\
= \sqrt{\frac{2gh}{\sin\alpha } \pab{\sin\alpha - \mu_{k} \cos\alpha}} = \sqrt{2gh \pab{1 - \mu_{k} \ctg\alpha}}.
\end{split}
\end{equation}
\subsection{Poprzez zasadę zachowania energii}
Praca siły tarcia jest wyrażona wzorem:
\begin{equation}
W_{T} = - F_{T} s = - \mu_{k} mg \cos\alpha \frac{h}{\sin\alpha} = - \mu_{k}mgh \ctg\alpha.
\end{equation}
Równanie na energię:
\begin{equation}
\begin{split}
\inv{2}mv^2 - mgh = - \mu_{k}mgh \ctg\alpha, \\
v^2 = 2gh\pab{1 - \mu_{k} \ctg\alpha}, \\
v = \sqrt{2gh\pab{1 - \mu_{k} \ctg\alpha}}.
\end{split}
\end{equation}

\end{document}