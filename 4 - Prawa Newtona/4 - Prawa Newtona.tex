\documentclass{article}
\usepackage[utf8]{inputenc}
\usepackage[T1]{fontenc}
\usepackage{babel}
\usepackage{float}
\usepackage{xcolor}  % kolory motywu
\usepackage{tikz}
\usetikzlibrary{angles}
\usetikzlibrary{quotes}
\usetikzlibrary{decorations.pathreplacing}
\usetikzlibrary{calligraphy}
\usetikzlibrary{arrows.meta}
\usetikzlibrary{calc}
\usepackage{pgfplots}
\usepackage{pgfplotstable}
\pgfplotsset{compat=1.9}
\usepackage{amsmath}  % równania
\usepackage{amssymb}
\usepackage{bbold}
\usepackage{physics2}  % pochodne, macierze itp
\usephysicsmodule{ab}
\usephysicsmodule{diagmat}
\usephysicsmodule{xmat}
\usephysicsmodule{nabla.legacy}
\usephysicsmodule{op.legacy}
\makeatletter
%\newcommand\vb[1]{\@ifstar\boldsymbol\mathbf{#1}}
\newcommand\vb[1]{\@ifstar\boldsymbol\mathbf{#1}}
\newcommand\va[1]{\@ifstar{\vec{#1}}{\vec{\mathrm{#1}}}}
\newcommand\vu[1]{%
	\@ifstar{\hat{\boldsymbol{#1}}}{\hat{\boldsymbol{#1}}}}
\makeatother
\usepackage{fixdif, derivative}  % pochodne
%\usepackage{mhchem}
\usepackage{siunitx}
\DeclareSIUnit{\nothing}{\relax}
\sisetup{parse-numbers = false}
\usepackage{booktabs}
\usepackage{tabularx}
\title{Prawa Newtona}
%\author{Rafał Staroszczyk}
\date{}

%\DeclareMathOperator{\Col}{Col}
%\DeclareMathOperator{\Nul}{Nul}
%\DeclareMathOperator{\arctg}{arctg}
%\DeclareMathOperator{\tgh}{tgh}


\setlength{\abovedisplayskip}{0pt}
\setlength{\belowdisplayskip}{0pt}
\setlength{\abovedisplayshortskip}{0pt}
\setlength{\belowdisplayshortskip}{0pt}

\newcommand{\inv}[1]{\frac{1}{#1}}

\usepackage{enumitem}
% Środowisko zadań
\usepackage{exercise}
\renewcommand{\ExerciseName}{Zadanie}
\renewcommand{\AnswerName}{Odpowiedź do zadania}
%\renewcommand{\ExerciseHeader}{\noindent{%
%\textbf{\large\ExerciseHeaderDifficulty\ExerciseName\ %
%\ExerciseHeaderNB\ExerciseHeaderTitle\ExerciseHeaderOrigin}}}
%\renewcommand{\AnswerHeader}{\noindent{\textbf{\AnswerName\ %
%\ExerciseHeaderNB}} }
\renewcommand{\theExePart}{(\alph{ExePart})\quad}
\renewcommand{\ExePartName}{}
\renewcommand{\ExePartHeader}{%
\indent{\ExePartHeaderDifficulty\ExePartName%
\ExePartHeaderNB\ExePartHeaderTitle}}
%\numberwithin{Exercise}{section}
%\numberwithin{Answer}{section}

\begin{document}
\maketitle
\section{Masa}
Masa od roku 1799 była zdefiniowana poprzez fizyczny obiekt, odważnik platynowy. Od roku 1889 podobnie przez inny odważnik platynowo irydowy. W roku 2019 znów przedefiniowano kilogram tak, aby nie odwoływał się do fizycznego obiektu, a do stałych fizycznych. Ta definicja jednak nie jest przydatna bez mechaniki kwantowej, więc przyjmę, że wciąz jest to obiekt fizyczny. \\
Aby porównać dwie masy można przykładowo wykorzystać wagę bezwładnościową. Dwa odważniki są zamocowane na końcach pręta, a w jego środku przyczepiona jest linka. Gdy zadziałamy siła na linkę i pręt sie nie obróci to możemy wywnioskować, że obie masy są identyczne z symetrii układu. Alternatywnie zamiast działać siłą na linkę, można umieścic układ w polu grawitacyjnym, korzystając z faktu, że każde ciało, niezależnie od masy, doświadcza takiego samego przyspieszenia. Jest to klasyczna waga grawitacyjna. Taki układ nie jest praktyczny, aby odmierzyć dowolną wielkość masy, ale wystarcza do analizy fizycznej.

\section{Treść Praw}
\begin{center}
\begin{enumerate}
\item Jeżeli na ciało nie działają siły zewnętrzne lub działające siły równoważą się, to ciało pozostaje w spoczynku lub porusza się ruchem jednostajnym prostoliniowym. 
\begin{equation}
\vb{F}_{wyp} = 0 \iff \vb{v} = \text{const.}
\end{equation}
\item Przyspieszenie ciała jest wprost proporcjonalne do siły wypadkowej, ma również ten sam kierunek co siła, natomiast jest odwrotnie proporcjonalne do masy ciała. Możemy to zapisać w formie równania: 
\begin{equation}
\vb{a} = \frac{\vb{F}_{wyp}}{m}.
\end{equation}
Lub inaczej: 
\begin{equation}
\vb{F}_{wyp} = \sum \vb{F} = m\vb{a}.
\end{equation}
\item Jeżeli jedno ciało działa na drugie pewną siłą, to drugie ciało działa na pierwsze siłą o takim samym kierunku i wartości, lecz przeciwnym zwrocie. Matematycznie, jeżeli ciało A działa siłą $\vb{F}$ na ciało B, wówczas jednocześnie ciało B, działa na ciało A siłą $-\vb{F}$. Wektorowo można to zapisać jako 
\begin{equation}
\vb{F}_{AB} = -\vb{F}_{BA}.
\end{equation}
\end{enumerate}
\end{center}

\section{Układ inercyjny}
\begin{center}
Jeżeli w jakimś układzie spełniona jest pierwsza zadana dynamiki Newtona, to układ ten jest układem inercjalnym. Dowolny inny układ, który porusza się ruchem jednostajnym prostoliniowym i się nie obraca względem pewnego układu inercjalnego również jest inercjalny. Nie ma jednego uprzywilejowanego układu inercjalnego i prawa fizyki działają w ten sam sposób w każdym z nich. 
\end{center}

\section{Siła}
Siła jest wielkością wektorową, zdefiniowaną poprzez drugą zasadę dynamiki Newtona. Jednostką siły jest niuton $\unit{\newton}=\unit{\kilo\gram\metre\per\second\squared}$, który jest taką siła, aby ciału o masie \qty{1}{\kg} nadać przyspieszenie \qty{1}{\m/\second\squared}. Bardziej praktycznym sposobem pomiaru siły jest pomiar pośredni wykorzystujący jakieś odkształcenie pod wpływem siły, na przykład rozciągająca się sprężyna, która w zakresie sprężystym w przybliżeniu spełnia zależność liniową zwaną prawem Hooke'a:
\begin{equation}
F = -kx,
\end{equation}
gdzie $k$ to stała opisująca sprężynę, a $x$ to wychylenie z położenia równowagi w kierunku osi sprężyny. Minus oznacza, że działająca siła będzie miała przeciwny zwrot do wychylenia.

\section{Inna forma drugiej zasady dynamiki Newtona}
Prędkość jest zdefiniowana jako $\vb{v} = \odv{\vb{r}}{t}$, a przyspieszenie $\vb{a} = \odv{\vb{v}}{t}$. Przyjmując, że masa jest stała, co jest prawdziwe w mechanice klasycznej:
\begin{equation}\label{eq:Fdpdt}
\vb{F} = m\vb{a} = m\odv{\vb{v}}{t} = \odv{(m\vb{v})}{t} = \odv{\vb{p}}{t},
\end{equation}
gdzie $\vb{p} = m\vb{v}$ jest pędem. Taka definicja będzie przydatna w późniejszych wzorach. Jest ona też bardziej ogólna, ponieważ uwzględnia ciała bezmasowe, które mają pęd, tj. fotony oraz jest to wzor wykorzystywany w szczególnej teorii względności.

\section{Alternatywna definicja siły}
Powyżej przedstawiono definicję siły poprzez drugą zasadę Newtona. Przy takiej definicji siła grawitacji wyrażona jest wzorem:
\begin{equation}\label{eq:grawitacja}
F = G\frac{m_{1}m_{2}}{r^{2}},
\end{equation}
gdzie $G \approx \qty{6,67.10^{-11}}{\m^{3}.\kg^{-1}.\s^{-2}}$ to stała grawitacji. Alternatywnie można zdefiniować siłe za pomocą tego prawa powszechnego ciążenia:
\begin{equation}
\tilde{F} = \frac{m_{1}m_{2}}{r^{2}}.
\end{equation}
Jak będzie wyglądać druga zasada dynamiki Newtona?

\section{Siły rzeczywiste a pozorne}
W fizyce wyróżnia się cztery oddziaływania:
\begin{enumerate}
\item Oddziaływanie grawitacyjne;
\item Oddziaływanie elektromagnetyczne;
\item Oddziaływanie silne;
\item Oddziaływanie słabe.
\end{enumerate}
Oddziaływania silne i słabe występują jedynie w skali jąder atomowych i podobnych cząstek. Każdą z doświadczanych przez nas sił można przypisać albo do grawitacji, albo elektromagnetyzmu. Przykładowo siła nacisku oraz reakcji podłoża jest związana z odpychaniem się naładowanych jąder i elektronów w ciałach. Jeżeli jakiejś sile można przypisać jedno z oddziaływań podstawowych to jest to siła rzeczywista, jeśli nie jest to siła pozorna wynikająca z wyboru nieinercjalnego układu odniesienia. Układy nieinercjalne to takie, które przyspieszają względem układów inercjalnych, albo poprzez zmianę wartości prędkości, albo kierunku, albo obydwa jednocześnie. \par
Siła związana z przyspieszeniem układu to siła bezwładności dana wzorem:
\begin{equation}
\vb{F}_{B} = -m\vb{a}.
\end{equation}
Jest ona wybrana w taki sposób, aby rozwiązanie było jednakowe dla układów inercjalnych i nieinercjalnych.

\section{Problem grawitacji w układach inercjalnych}
W klasycznej teorii grawitacji Newtona jest ona siłą wyrażoną wzorem \eqref{eq:grawitacja}. Układ inercjalny (pomijając niewielkie wpływy związanie z obrotem planety itp.) jest stacjonarny względem ziemii lub porusza się ruchem jednostajnym prostoliniowym względem niej. \par
W relatywistycznej teorii grawitacji Einsteina jest to zakrzywienie czasoprzestrzeni, przez co nie jest to siła, a układy inercjalne to takie w spadku swobodnym. \par
W pewnym sensie więc siła grawitacji jest siłą bezwładnośći związaną z przyspieszającym układem odniesienia.

\section{Układ wielu cząstek}
Niech na każdą cząstkę działają siły między sobą nawzajem oraz siła zewnętrzna. Wykorzystując wtedy \eqref{eq:Fdpdt} mamy:
\begin{equation}
\odv{\vb{p}_{k}}{t} = \sum_{j \neq k} \vb{F}_{kj} + \vb{F}_{k}^{zew}.
\end{equation}
Pochodna całkowitego pędu:
\begin{equation}
\odv{\vb{P}}{t} = \sum_{k} \odv{\vb{p}_{k}}{t} = \sum_{k} \sum_{j \neq k} \vb{F}_{kj} + \sum_{k} \vb{F}_{k}^{zew}.
\end{equation}
W wyrażeniu $\sum_{k} \sum_{j \neq k} \vb{F}_{kj}$ występuje zarówno $\vb{F}_{kj}$ jak i $\vb{F}_{jk}$, można więc ograniczyć sumę do:
\begin{equation}
\odv{\vb{P}}{t} = \sum_{k} \sum_{j > k} \bab{\vb{F}_{kj} + \vb{F}_{jk}} + \vb{F}^{zew}.
\end{equation}
Z trzeciej zasady dynamiki mamy $\vb{F}_{kj} + \vb{F}_{jk} = 0$, więc:
\begin{equation}
\odv{\vb{P}}{t} = \vb{F}^{zew}.
\end{equation}
Stąd układ takich ciał zachowuje się jak pojedyncze ciało o masie $M = \sum_{k} m_{k}$.

Zapiszmy położenia wielu cząstek jako
\begin{equation}
\vb{r}_{k} = \vb{R} + \vb{r'}_{k}.
\end{equation}
Wtedy ich prędkości to odpowiednio
\begin{equation}
\vb{v}_{k} = \vb{V} + \vb{v'}_{k},
\end{equation}
a pędy
\begin{gather}
m_{k}\vb{v}_{k} = m_{k}\vb{V} + m_{k}\vb{v'}_{k}, \\
\vb{p}_{k} = m_{k}\vb{V} + \vb{p'}_{k}.
\end{gather}
Pęd całkowity ma postać:
\begin{equation}
\begin{split}
\vb{P} &= \sum_{k} \vb{p}_{k} = \sum_{k} m_{k}\vb{v}_{k} = \\
&= \sum_{k} m_{k}\vb{V} + \sum_{k} m_{k}\vb{v'}_{k}.
\end{split}
\end{equation}
Jako, że wektor $\vb{R}$ możemy wybrać dowolnie możemy wybrać taki, aby ${\sum_{k} m_{k}\vb{v'}_{k} = 0}$. Narzuca to warunek:
\begin{equation}
\begin{split}
\sum_{k} m_{k}\vb{v}_{k} = \sum_{k} m_{k}\vb{V} \\
\vb{V} = \inv{\sum_{k} m_{k}} \sum_{k} m_{k}\vb{v}_{k} \\
\odv{}{t} \vb{R} = \odv{}{t} \inv{M} \sum_{k} m_{k}\vb{r}_{k}.
\end{split}
\end{equation}
Mamy stąd, że:
\begin{align}
M &= \sum_{k} m_{k}, & \vb{R} &= \inv{M} \sum_{k} m_{k}\vb{r}_{k}, & \vb{V} &= \odv{\vb{R}}{t}, & \vb{P} &= M\vb{V}.
\end{align}
Wektor $\vb{R}$ jest wektorem wodzącym środka masy układu. Pozwala to rozdzielić opis zjawiska na ruch makroskopowy, opisany całkowitą siłą zewnętrzną, całkowitą masa i podobnymi wielkościami, oraz na ruch wewnątrz ciała.

\section{Układ dwóch cząstek}
Dla dwóch punktów materialnych w położeniach $\vb{r}_{1}$ oraz $\vb{r}_{2}$ możemy wyróżnić ruch na ruch układuoraz ruch względny cząstek według poprzedniej analizy, ale zamiast wyznaczać położenia $\vb{r'}_{k}$ względem środka masy wprowadzamy:
\begin{equation}
\vb{r} = \vb{r}_{1} - \vb{r}_{2}.
\end{equation}
Jest to lepszy wybór, ponieważ wiele sił zależy jedynie od odległości między cząsteczkami. \par
Jeżeli na cząstkę pierwszą działa siła ${\vb{F}_{12} = m_{1}\vb{{a}}_{1}}$, a na drugą ${\vb{F}_{21} = m_{2}\vb{{a}}_{2}}$, przy czym ${\vb{F}_{12} = -\vb{F}_{21}}$ to mamy:
\begin{equation}
m_{1}\vb{a}_{1} = -m_{2}\vb{a}_{2} \Rightarrow \vb{a}_{2} = -\frac{m_{1}}{m_{2}} \vb{a}_{1}.
\end{equation}
Przyspieszenie względne cząstek to druga pochodna ${\vb{r}}$:
\begin{equation}
\vb{a} = \vb{a}_{1} - \vb{a}_{2} = \pab{1 + \frac{m_{1}}{m_{2}}} \vb{a}_{1} = \frac{m_{1} + m_{2}}{m_{1}m_{2}}\vb{F}_{12} = \frac{\vb{F}_{12}}{\mu},
\end{equation}
gdzie ${\mu = \frac{m_{1}m_{2}}{m_{1} + m_{2}}}$ to masa zredukowana. Spełnia ona również ${\inv{\mu} = \inv{m_{1}} + \inv{m_{2}}}$.

\section{Siła sprężystości}
W zakresie elastycznym materiały wykazują liniową zależność między siłą a wychyleniem z punktu równowagi. Siła ta jest przeciwnie skierowana względem wychylenia i jest opisana wzorem:
\begin{equation}
\vb{F} = - k \vb{x},
\end{equation}
gdzie $k$ jest stałą sprężystości, a $\vb{x}$ to wychylenie z punktu równowagi. \par
Sprężyny można ze sobą łączyć szeregowo lub równolegle. W przypadku równoległym rozciągnięcie dwóch sprężyn jest takie same, a siły się dodają. Mamy więc:
\begin{gather}
\vb{F} = - k\vb{x} = \vb{F}_{1} + \vb{F}_{2} = - k_{1}\vb{x} - k_{2}\vb{x} = - (k_{1} + k_{2})\vb{x} \\
k = k_{1} + k_{2}.
\end{gather}
Dla połączenia sprężyn szeregowego mamy wspólną siłę, a rozciągnięcia się dodają:
\begin{gather}
\vb{F} = - k_{1} \vb{x}_{1} = - k_{2} \vb{x}_{2} \quad\Rightarrow\quad x_{1} = - \frac{\vb{F}}{k_{1}}; \quad x_{2} = - \frac{\vb{F}}{k_{2}} \\
\vb{F} = - k \vb{x} = - k \pab{\vb{x_{1}} + \vb{x_{2}}} = - k \pab{-\frac{\vb{F}}{k_{1}} - \frac{\vb{F}}{k_{2}}} \quad\Rightarrow\quad \inv{k} = \inv{k_{1}} + \inv{k_{2}}.
\end{gather}
\end{document}